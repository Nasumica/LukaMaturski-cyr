\subsection{Кватерниони}

\def\uv{{u}}
\def\vp{{v}}
\def\norm#1{{\vert#1\vert}}
\def\con#1{{\bar#1}}

Попут скупа комплексних бројева $\Cset$ који представља објекте у равни, односно, у~\idx{2D} простору,
скуп {\sl кватерниона\/}\index{кватернион}~$\Hset$ представља објекте у~\idx{3D} простору. 
Први их је описао 1843.\ године ирски математичар
\idx{Хамилтон} (William Rowan Hamilton), те њему у част и ознака скупа
$\Hset$.
У информатици су неизбежни део свега што се дешава у~3D:
навигација авиона, подморница, ракета, сателита,
небеска и квантна механика, роботика, игре, графика,~\dots

\medskip

Кватернион $q\in\Hset$ може бити представљен као \idx{збир}
\begin{equation}
    q=s+\vp
\end{equation}
који се састоји од {\sl скаларног\/}\index{скалар} дела $s\in{\mathbb R}$ и {\sl векторског\/}\index{вектор} дела 
$\vp\in\Rset^3$, где је
\begin{equation}
    \vp=xi+yj+zk
\end{equation}
3D вектор са координатама $(x,y,z)$, 
а где су $i$\index{i@$i$}, $j$\index{j@$j$} и $k$\index{k@$k$} јединични вектори\index{јединични вектор} 
по $x$\idxaxis x, $y$\idxaxis y и $z$\idxaxis z оси, за које важи
\begin{equation}\label{eq:qunits}
    i^2=j^2=k^2=ijk=-1,\quad
    ij=k,\quad jk=i,\quad ki=j. 
\end{equation}

\danger 
У скупу кватерниона $\Hset$ за операцију множења, уопштено, не важи {\sl закон комутације}\index{комутативност}:
$ji=-ij=-k$, $kj=-jk=-i$, $ik=-ki=-j$.
Ово је логично кад се сетимо да и код
{\sl Рубикове коцке\/}\index{Рубикова коцка} 
најчешће није свеједно којим редоследом окрећемо странице.
Важи {\sl \idx{асоцијативност}}: $(p\cdot q)\cdot r=p\cdot(q\cdot r)$.

\medskip

Да би $q=s+\vp$ био {\sl прави\/} кватернион, мора бити $\vp\ne0$, иначе је $q$ обичан реалан број,
када се примењују операције и функције из скупа $\Rset$.
Ако одредимо апсолутну вредност\index{апсолутна вредност $\vert x\vert$} кватерниона\index{ламбда $(\lambda)$}\index{ро $(\rho)$}
$$
\lambda = \norm \vp = \sqrt{\mathstrut x^2+y^2+z^2},\qquad
\rho = \norm q = \sqrt{\mathstrut s^2+\lambda^2}
$$
која се зове {\sl \idx{норма}\/}, одредимо
{\sl јединични вектор\/} ({\sl unit\/}) векторског дела кватерниона
$$
\uv = \frac \vp\lambda, 
$$
који се зове {\sl версор\/}\index{версор}
и где је по дефиницији\footnote{У 
скупу $\Hset$, $\sqrt{-1}$ има бесконачно решења:
сваки кватернион који се налази на {\sl јединичној сфери\/}
($s=0\land x^2+y^2+z^2=1$) је решење, односно, сваки версор.} 
$\norm\uv=1$ и  $\uv^2=-1$,
као и угао оријентације\index{фи $(\varphi)$}
$$
\varphi = \arccos\left( \frac s\rho \right),
$$
можемо добити запис кватерниона\index{поларни запис}
\begin{equation}
q
= s + \lambda\uv
=\rho\,(\cos\varphi + \uv\sin\varphi)
=\rho\,\e^{\uv\varphi}.
\end{equation}
Из свега овога се може добити
\begin{equation}\index{ln@$\ln$}
    \okvir{\ln(q)  = \ln\rho + \uv\varphi}
\end{equation}
\centerline{и}
\begin{equation}\index{exp@$\exp$}
    \okvir{\exp(q) = \e^s \left( \cos\lambda + \uv\sin\lambda \right)}\rlap{.}
\end{equation}

\medskip

\danger
Остале операције и функције нису тема овог рада, али сабирање и одузимање
је уобичајено, код множења треба обратити пажњу на формулу \eqref{eq:qunits} и
комутативност, а \idx{реципрочна вредност} је
$q^{-1}=\con q/\rho^2$, где је $\con q=s-\vp$, {\sl конјугована\/} вредност\index{конјугована вредност $(\con z)$}. 
Тригонометријске и хиперболичне функције се могу изразити помоћу експоненцијалне,
а њихове инверзне помоћу логаритамске функције.

