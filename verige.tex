\documentclass[12pt, twoside, a4paper]{article}

\usepackage[margin=1truein]{geometry}
\usepackage{amsmath}
\usepackage{graphicx}
\usepackage{tabularx}
%\usepackage[backend=biber]{biblatex}

\input serbian

\def\vr#1{\ifcase#1\relax\or
  ve\-ri{\zv}\-ni raz\-lo\-mak\or
  ve\-ri{\zv}\-nog raz\-lom\-ka\or
  ve\-ri{\zv}\-nom raz\-lom\-ku\or
  ve\-ri{\zv}\-ni raz\-lo\-mak\or
  ve\-ri{\zv}\-ni raz\-lom\-{\cv}e\or
  ve\-ri{\zv}\-nim raz\-lom\-kom\or
  ve\-ri{\zv}\-nom raz\-lom\-ku\fi}
\def\Vr#1{\ifcase#1\relax\or
  Ve\-ri{\zv}\-ni raz\-lo\-mak\or
  Ve\-ri{\zv}\-nog raz\-lom\-ka\or
  Ve\-ri{\zv}\-nom raz\-lom\-ku\or
  Ve\-ri{\zv}\-ni raz\-lo\-mak\or
  Ve\-ri{\zv}\-ni raz\-lom\-{\cv}e\or
  Ve\-ri{\zv}\-nim raz\-lom\-kom\or
  Ve\-ri{\zv}\-nom raz\-lom\-ku\fi}


\renewcommand*\contentsname{Sadr{\zv}aj}

\usepackage{titlesec}
\newcommand{\sectionbreak}{\clearpage
  \vskip 0.75in plus 0.25in minus 0.25in\relax}
%\newcommand{\subsectionbreak}{\clearpage}

\def\navod#1{\relax,\kern-0.06667em,\relax#1\relax``\relax}

\def\A{p} \def\B{q}

\begin{document}

\begin{titlepage}
\setcounter{page}{0}
\thispagestyle{empty}
\begin{center}
\large
Gimnazija \navod{Bora Stankovi{\cc}}\\
Ni{\sv}, Srbija\\
\vfill
\LARGE
\textbf{MATURSKI RAD}\\
\Large
\bigskip
Predmet: Matematika\\
Tema: Veri{\zv}ni razlomci\\
\vfill\vfill
\begingroup
\large
\setbox0=\hbox{Luka Ne{\sv}i{\cc}, IV/\,6}%
\begin{tabularx}{\textwidth}{lXl}
\rule{\wd0}{0pt}&&\rule{\wd0}{0pt}\\
U{\cv}enik:&&Profesor:\\
\copy0&&Nenad Toti{\cc}\\
%\noalign{\bigskip}
%\rule{\wd0}{0.2pt}&&\rule{\wd0}{0.2pt}\\
\end{tabularx}
\endgroup
\vskip 2truecm
\normalsize
\the\year.
\end{center}
\end{titlepage}

\let\ds=\displaystyle


\tableofcontents

\section{Uvod}

\subsection{Definicija}

Izraz oblika
$$
x=b_0+\cfrac{a_1}{b_1+\cfrac{a_2}{b_2+\cfrac{a_3}{b_3+\cfrac{a_4}{b_4+\ddots}}}}
$$
u matematici se zove {\sl\vr1\/} (re{\cv} {\it verige\/} zna{\cv}i {\it lanac\/}).
Brojilac ovakvog raz\-lom\-ka je broj, a imenilac mo{\zv}e biti broj, obi{\cv}an razlomak ili
veri{\zv}ni razlomak.

\subsection{Notacija}

Iako je standardni zapis \vr2 jasan i onima koji nisu mnogo upoznati sa {\nj}ima,
{\cv}esto se, {\sv}to zbog u{\sv}tede prostora, {\sv}to zbog jednostavnijeg i kra{\cc}eg pisa{\nj}a,
koriste i druga{\cv}ije notacije.

\medskip

Prva formalna skra{\cc}ena notacija poti{\cv}e sa po{\cv}etka 17.~veka
koju je koristio italijanski matemati{\cv}ar Kataldi (Pietro Antonio Cataldi)
$$
x=b_0\bullet 
\& \frac{a_1}{b_1\bullet} 
\& \frac{a_2}{b_2\bullet} 
\& \frac{a_3}{b_3\bullet} 
\& \frac{a_4}{b_4\bullet} 
\& \cdots
$$
gde simbol `$\bullet$' ozna{\cv}ava mesto na kome se nalazi ostatak izraza, a simbol `$\&$' predstav{\lj}a
znak za sabira{\nj}e `$+$'.
Od ovog zapisa poti{\cv}e i savremena skra{\cc}ena notacija \vr2
$$
x=b_0+
\frac{a_1}{b_1+}\,
\frac{a_2}{b_2+}\,
\frac{a_3}{b_3+}\,
\frac{a_4}{b_4+}\,
\cdots
$$
gde spu{\sv}teni znak `$+$' ozna{\cv}ava mesto gde {\cc}e se {\sl ugnezditi\/} ostatak izraza.
Ponekad se znak `$+$' pi{\sv}e van razloma{\cv}ke crte, ali ostaje {\sl poravnat\/} sa imeniocima
\def\dop#1{\mathbin{{\atop#1}}}
\def\bplus{\dop+}
\def\fplus#1#2{\bplus\frac{#1}{#2}}
$$
x=b_0+
\frac{a_1}{b_1}
\fplus{a_2}{b_2}
\fplus{a_3}{b_3}
\fplus{a_4}{b_4}
\bplus{\atop\cdots}
$$
Nema{\cv}ki matemati{\cv}ar Pringzhajm (Alfred Pringsheim) je koristio slede{\cc}u notaciju
\def\dvert{{\atop {{\big |}\!}}\!}
\def\nvert{\!{{\!{\big |}} \atop}}
\def\pring#1#2{\dvert\frac{\,#1\,}{\,#2\,}\nvert}
$$
x=b_0
+\pring{a_1}{b_1}
+\pring{a_2}{b_2}
+\pring{a_3}{b_3}
+\pring{a_4}{b_4}
+\cdots
$$

\medskip

\def\K{\mathop{\vcenter{\hbox{\rm\huge K}}}\limits}
\def\Ki{\K_{i=1}}
\def\Kinf#1#2{\Ki^\infty\displaystyle{\frac{#1}{#2}}}

Poput simbola koji se koriste za sumu `$\rm\Sigma$' ili proizvod `$\rm\Pi$', 
Gaus (Carl Friedrich Gauss) je smislio, verovatno, najpogodniji na{\cv}in za predstav{\lj}a{\nj}e
veri{\zv}nih razlomaka
$$
%\fbox{$\displaystyle
x=b_0+\Kinf{a_i}{b_i}
%$}
$$
gde simbol `K' poti{\cv}e od nema{\cv}ke re{\cv}i za {\sl prekinuti lanac\/} (Kettenbruch). 


\section{Teorija}

\subsection{Vrste}

U zavisnosti od toga da li imaju koana{\cv}an ili beskona{\cv}an broj {\cv}lanova,
veri{\zv}ni razlomci mogu biti {\sl kona{\cv}ni\/} ili {\sl beskona{\cv}ni}.
Vrednost kona{\cv}nih ralomaka mo{\zv}e biti izra{\cv}unati sa apsolutnom ta{\cv}no{\sv}{\cc}u,
dok za beskona{\cv}ne vrednost mo{\zv}e biti izra{\cv}unata kao
$$x=b_0+\lim_{n\to\infty}\Kinf{a_i}{b_i},$$
ili pribli{\zv}na numeri{\cv}ka vrednost.
Ako imaju racionalne {\cv}lanove, kona{\cv}ni razlomci su racionalni, a beskona{\cv}ni
iracionalni brojevi, sem specijalnih slu{\cv}ajeva kao {\sv}to je
$$
\Kinf32=1.
$$

\medskip

\Vr1 kome su svi brojioci $a_i=1$ se zove {\sl obi{\cv}an}
$$
x=b_0+\Kinf1{b_i}.
$$

\subsection{Konvergent}

Veri{\zv}ni razlomak izra{\cv}unat sa prvih $n$ {\cv}lanova
$$
x_n=b_0+\Ki^n{a_i\over b_i}={\A_n\over \B_n}
$$
se zove $n$-ti {\sl konvergent} i mo{\zv}e biti predstav{\lj}en kao obi{\cv}an razlomak $x_n=\A_n/\B_n$.
Va{\zv}i formula
$$
\A_{n-1} \B_n - \A_n \B_{n-1} = \prod_{i=1}^n(-a_i)
$$
koja se zove {\sl formula determinante}.


\subsection{Rekurentna formula}

Parcijalni de{\lj}enik $p_n$ i delilac $q_n$ konvergenata $x_n=\A_n/\B_n$, mo{\zv}e se izra{\cv}unati
re\-ku\-rent\-nom formulom
\begin{alignat*}{4}
%&\A_{-1}&&=1&&\B_{-1}&&=0\\
&\A_0&&=b_0&&\B_0&&=1\\
&\A_1&&=b_1p_0+a_1&&\B_1&&=b_1\\
&\>\vdots&&&&\>\vdots\\
&\A_n&&=b_n\A_{n-1}+a_n\A_{n-2}\qquad&&\B_n&&=b_n\B_{n-1}+a_n\B_{n-2}
\end{alignat*}
gde se, poput Fibona{\cv}ijevih brojeva, konvergent ra{\cv}una pomo{\cc}u prethodna dva.

\subsection{Standardna evaluacija}

Kona{\cv}ni \vr1 mo{\zv}e biti izra{\cv}unat i standardnim na{\cv}inom
\begin{align*}
&x\leftarrow0\\
&i=n,n-1,\ldots,1:\\
&\qquad x\leftarrow a_i/(b_i+x)\\
&x\leftarrow b_0+x
\end{align*}
od dna ka vrhu razlomka. Tim na{\cv}inom se ne dobijaju konvergenti, ve{\cc} samo kona{\cv}na vrednost.

\subsection{Transformacije}

\Vr1 mo{\zv}e biti {\sl pomno{\zv}en\/} nizom konstanti $c$ na slede{\cc}i na{\cv}in
$$
b_0+\cfrac{a_1}{b_1+\cfrac{a_2}{b_2+\cfrac{a_3}{b_3+\cfrac{a_4}{b_4+\ddots}}}}
=
b_0+\cfrac{a_1c_1}{b_1c_1+\cfrac{a_2c_1c_2}{b_2c_2+\cfrac{a_3c_2c_3}{b_3c_3+\cfrac{a_4c_3c_4}{b_4c_4+\ddots}}}}
$$
Svaki \vr1 mo{\zv}e biti pretvoren u obi{\cv}ni \vr1 slede{\cc}om transformacijom
$$
b_0+\Kinf{a_i}{b_i} 
= 
b_0+\Kinf1{b_ic_i}, \quad\text{gde je}\quad
c_1={1\over a_1}, \ldots, 
c_n={1\over a_n c_{n-1}}, \ldots
$$
Tako{\dj}e va{\zv}i i transformacija
$$
b_0+\Kinf{a_i}{b_i} 
= 
b_0+\Kinf{a_i d_i}1, \quad\text{gde je}\quad
d_1={1\over b_1}, \ldots, 
d_n={1\over b_n b_{n-1}}, \ldots
$$

\subsection{Racionalizacija}

Ako je $x$ neki realni broj, on mo{\zv}e biti pretvoren u \vr1
za {\zv}e{\lj}enim brojem {\cv}lanova $n$ i
sa {\zv}e{\lj}enom
ta{\cv}no{\sv}{\cc}u $\varepsilon$ na slede{\cc}i na{\cv}in
\begin{align*}
&b_0\leftarrow[x]\\
&x\leftarrow x-b_0\\
&i=1,2,\ldots,n\land |x|>\varepsilon:\\
&\qquad x\leftarrow 1/x\\
&\qquad b_i\leftarrow[x]\\
&\qquad x\leftarrow x-b_i
\end{align*}
gde $[x]$ zna{\cv}i {\sl celobrojna vrednost}. To mo{\zv}e biti ili obi{\cv}no {\sl odseca{\nj}e\/} decimala
ili zaokru{\zv}ena vrednost broja na ceo broj.

Metoda racionalizacije mo{\zv}e biti iskori{\sv}tena i za {\sl upor{\sv}{\cc}ava{\nj}e\/} razlomka.
U slu{\cv}aju kada je $x$ razlomak kome
brojilac i imenilac imaju mnogo cifara, mo{\zv}e biti pretvoren u pribli{\zv}nu vrednost, 
razlomkom sa ma{\nj}e cifara.


\section{Funkcije}

Ojler (Leonhard Euler) je dokazao da red koji mo{\zv}e biti napisan kao
\begin{align*}
x & =c_0(1+c_1(1+c_2(1+c_3(\cdots)))),\\
\intertext{odnosno, kao}
&=c_0+c_0c_1+c_0c_1c_2+c_0c_1c_2c_3+\cdots,\\
\intertext{onda je on jednak \vr3}
&=\cfrac{c_0}{1-\cfrac{c_1}{1+c_1-\cfrac{c_2}{1+c_2-\cfrac{c_3}{1+c_3-\ddots}}}}\\
&={c_0\over1+}\Kinf{-c_i}{1+c_i}.
\end{align*}
Pomo{\cc}u ove jednakosti mogu{\cc}e je pretvoriti funkciju predstav{\lj}enu
Tejlorovim (Brook Taylor) ili 
Maklorenovim (Colin Maclaurin) redom u \vr1.

U nastavku ovog poglav{\lj}a bi{\cc}e prikazani veri{\zv}ni razlomci za izra{\cv}unava{\nj}e
{\sl stan\-dard\-nih\/} matemati{\cv}kih funkcija.

\subsection{exp}

\def\d{{\cdot}}
Iz Maklorenovog reda za izra{\cv}unava{\nj}e funkcije $e^x$
\begin{align*}
\exp x  &= \sum_{i=0}^\infty{x^i\over i!}\\
&= 1 + {x^1\over1!} + {x^2\over2!} + {x^3\over3!} + {x^4\over4!} + \cdots\\
\noalign{\smallskip}
%&= 1 + {x\over1}\left(1+{x\over 2}\left(1+{x\over 3}\left(1+{x\over 4}\left(\cdots\right)\right)\right)\right)\\
&= 1 + {x\over1}+{x\over1}\d{x\over2}+{x\over1}\d{x\over2}\d{x\over3}+{x\over1}\d{x\over2}\d{x\over3}\d{x\over4}+\cdots\\
\intertext{koriste{\cc}i Ojlerovu jednakost ($c_0=1, c_i=x/i$)
i transformaciju jednakosti, dobija se \vr1}
\exp x &= \cfrac{1}{1-\cfrac{x}{1+x-\cfrac{1x}{2+x-\cfrac{2x}{3+x-\cfrac{3x}{4+x-\ddots}}}}}\\
&= \frac{1}{1-}\, \frac{x}{1+x+}  \Kinf{-ix}{i+1+x}.
\end{align*}

\subsection{ln}

Sli{\cv}no eksponencijalnoj funkciji, iz Tejlorovog reda se mo{\zv}e dobiti veri{\zv}ni razlomak
za izra{\cv}unava{\nj}e vrednosti logarritamske funkcije, s tom razlikom da se izra{\cv}unava{\nj}e vr{\sv}i {\sl u~okolini\/} ta{\cv}ke 1,
odnosno bi{\cc}e izra{\cv}unata vrednost funkije za $x+1$
\def\lnp{\mathop{\rm ln^*}}
\begin{align*}
\lnp x&=\ln(x+1)\\
&=\cfrac{x}{1-0x+\cfrac{1^2x}{2-1x+\cfrac{2^2x}{3-2x+\cfrac{3^2x}{4-3x+\ddots}}}}\\
&=\frac{x}{1+} \Kinf{i^2x}{i+1-ix}.
\end{align*}
Kako funkcija konvergira samo za $x<1$, pre izra{\cv}unava{\nj}a potrebno je {\sl svesti\/} argument
na odgovaraju{\cc}i opseg
$$
\ln x=
\begin{cases}
\lnp(x-1)&\text{za $x<1$},\\
-\lnp(1/x-1)&\text{za $x\ge1$}.
\end{cases}
$$

\subsection{sin}

\begin{align*}
\sin x
&=\cfrac{x}{1+\cfrac{x^2}{2\cdot3-x^2+\cfrac{2\cdot3x^2}{4\cdot5-x^2+\cfrac{4\cdot5x^2}{6\cdot7-x^2+\ddots}}}}\\
\noalign{\smallskip}
&=\frac{x}{1+}\,\frac{x^2}{2\cdot3-x^2+}\Kinf{2i(2i+1)\cdot x^2}{(2i+2)(2i+3)-x^2}
\end{align*}

\subsection{cos}

\begin{align*}
\cos x
&=\cfrac{1}{1+\cfrac{x^2}{1\cdot2-x^2+\cfrac{1\cdot2x^2}{3\cdot4-x^2+\cfrac{3\cdot4x^2}{5\cdot6-x^2+\ddots}}}}\\
\noalign{\smallskip}
&=\frac{1}{1+}\,\frac{x^2}{1\cdot2-x^2+}\Kinf{(2i-1)2i\cdot x^2}{(2i+1)(2i+2)-x^2}
\end{align*}


\subsection{arctan}

Ojler je razvio formulu za izra{\cv}unava{\nj}e vrednosti
$\arctan(y/x)$ \vr6
\begin{align*}
\arctan{y\over x} &=
\cfrac{y}{1x+\cfrac{(1y)^2}{3x+\cfrac{(2y)^2}{5x+\cfrac{(3y)^2}{7x+\ddots}}}}\\
&={y\over x+}\Kinf{(iy)^2}{(2i+1)x}
\end{align*}
Ako je 
$$z=x+iy=\rho\,(\cos\theta+i\sin\theta)=\rho\,e^{i\theta},$$ 
ta{\cv}ka u ravni predstav{\lj}ena kompleksim brojem, 
formula {\cc}e iz\-ra\-{\cv}u\-na\-ti $\theta=\arg z$,
ugao koji zaklapa vektor $(0,0)$--$(x,y)$ sa $x$-osom.
$$
\arg(x+iy)=
\begin{cases}
\arctan(y/x)&\text{za $x>0$},\\
\arctan(y/x)+\pi&\text{za $x<0$ i $y\ge0$},\\
\arctan(y/x)-\pi&\text{za $x<0$ i $y<0$},\\
+\pi/2&\text{za $x=0$ i $y>0$},\\
-\pi/2&\text{za $x=0$ i $y<0$},\\
0&\text{za $x=0$ i $y=0$}.
\end{cases}
$$

\def\arccot{\mathop{\rm arccot}}
Ostale inverzne trigonometrijske funkcije mogu biti izra{\cv}unate pomo{\cc}u funkcije $\arctan$
$$
\arcsin x = \arctan{x\over \sqrt{1-x^2}},\qquad
\arccos x = \arctan{\sqrt{1-x^2}\over x},\qquad
\arccot x = \arctan{1\over x}.
$$

\iffalse

\subsection{Stepenova{\nj}e}

\def\rn#1{#1(2x^n+y)}
\def\rd#1{(m^2-#1^2n^2)y^2}
\begin{align*}
z^{m/n}&=(x^n+y)^{m/n}\\
&=x^m+
\cfrac{2x^m\cdot my}{\rn n-my+
\cfrac{\rd1}{\rn3+\cfrac{\rd2}{\rn5+\cfrac{\rd3}{\rn7+\ddots}}}}\\
&=x^m+{2x^m\cdot my \over \rn n-my+}\K_{i=1}^\infty{\rd i \over (2i+1)(2x^n+y)}
\end{align*}

\fi

\subsection{Kvadratni koren}

\begin{align*}
\sqrt z&=\sqrt{x^2+y}\\
&=x+
\cfrac{y}{2x+\cfrac{y}{2x+\cfrac{y}{2x+\cfrac{y}{2x+\ddots}}}}\\
&=x+\Kinf y{2x}
\end{align*}



\section{Konstante}

\subsection{Zlatna sredina $\varphi$}

Mo{\zv}da najpoznatiji \vr1 je
\begin{align*}
\varphi &=1+\Kinf11\\
&=1+\cfrac{1}{1+\cfrac{1}{1+\cfrac{1}{1+\cfrac{1}{1+\ddots}}}},\\
\intertext{Vidimo da se ispod prve razloma{\cv}ke crte opet nalazi $\varphi$, pa je}
\varphi&=1+\cfrac{1}{\varphi}.\\
\intertext{Re{\sv}e{\nj}e}
\varphi &= \frac{1+\sqrt5}{2} \approx 1.618034
\end{align*}
je poznato u matematici kao {\sl zlatni presek\/} ili {\sl zlatna sredina}.
Zbog svoje prirode, ovo je nasporije mogu{\cc}i kovergiraju{\cc}i \vr1,
odnosno, $\varphi$ predstav{\lj}a {\sl najiracionalniji\/} broj.
Izra\-{\cv}u\-na\-va\-ju{\cc}i uzastopne {\cv}lanove razlomka, $\varphi$ {\cc}e
dobijati vrednosti
$$
\varphi = \frac{1}{1},\frac{2}{1},\frac{3}{2},\frac{5}{3},\frac{8}{5},
\frac{13}{8},\frac{21}{13},\ldots
$$
odnosno, svaki konvergent {\cc}e biti jednak koli{\cv}niku dva uzastopna Fibona{\cv}ijeva (Leonardo Fibonacci) broja.

\bigskip

\Vr1 oblika
$$
\delta =k+\Kinf1k = k+{1\over \delta}={k+\sqrt{k^2+4}\over2}
$$
se zove {\sl metalna sredina}, gde se za $k=1$ dobija {\sl zlatna}, za $k=2$ {\sl srebrna},
za $k=3$ {\sl bronzana}, za $k=4$ {\sl bakarna}, \dots\ \
I ovde je $n$-ti konvergent koli{\cv}nik dva uzastopna\break $k$-Fi\-bo\-na\-{\cv}i\-je\-va broja za koje va{\zv}i
rekurentna formula
$$
F_n = k\cdot F_{n-1}+F_{n-2}.
$$ 

Veza izme{\dj}u nagiba logaritamske spirale $\mu$ i koeficijenta metalne sredine
$$
k=2\,\sinh\left(90^\circ\tan\mu\right).
$$
Spiralni kraci na{\sv}e galaksije {\sl Kumova slama}, {\cv}ine logaritamsku spiralu sa nagibom od oko 
$\mu\approx13^\circ$,
{\sv}to odgovara metalnoj sredini za 
$k\approx 43/58$,
tako da ne {\cv}ine zlatni presek. 
%Za zlatni presek mora biti $\mu=17.03239113^\circ$.

%\clearpage

\subsection{$\pi$}

Izra{\cv}unava{\nj}e broja $\pi$ je kroz istoriju uvek predstav{\lj}alo izazov.
Od davnina se koristio razlomak $22/7=3.\overline{142857}$ koji ima ta{\cv}nost 2 decimale
i razlomak $355/113\approx3.141593$ koji ima zadiv{\lj}uju{\cc}u ta{\cv}nost od 6 decimala.
Trenutno je poznato $202\,112\,290\,000\,000$ decimala.
Broj $\pi$ na 50 decimala je
$$
\pi=3.
1415926535\,
8979323846\,
2643383279\,
5028841971\,
6939937511\,
\ldots
$$

\medskip

Lajbnic (Gottfried Wilhelm Leibniz) je iz formule 
$${\pi\over 4}=1-{1\over3}+{1\over5}-{1\over7}+{1\over9}-\cdots$$
izveo formulu za izra{\cv}unava{\nj}e
broja $\pi$ \vr6
\begin{align*}
\pi
%\pi = \frac{4}{1}\fplus{1^2}{2}\fplus{3^2}{2}\fplus{5^2}{2}\fplus{7^2}{2}\bplus{\atop\cdots}
&= \cfrac{4}{1+\cfrac{1^2}{2+\cfrac{3^2}{2+\cfrac{5^2}{2+\ddots}}}}
=\frac{4}{1+}\Kinf{(2i-1)^2}2
\end{align*}
Na {\zv}alost, formula vrlo sporo konvergira broju $\pi$ i potrebno je izra{\cv}unati oko
$3\cdot10^n$ {\cv}lanova izraza da bi se dobilo $n$ ta{\cv}nih decimala.

\smallskip

Iako je $\pi=4\arctan(1)$, postoje formule koje br{\zv}e konvergiraju.
Najbr{\zv}a poznata formula za izra{\cv}unava{\nj}e broja $\pi$ \vr6 je
Ma{\cv}inova (John Machin) formula koja korsiti Ojlerovu formulu za $\arctan$
\begin{alignat*}{3}
&\pi
&&=16\arctan(1/5)
&&-4\arctan(1/239)\\
&{}
&&= \cfrac{16}{u+\cfrac{1^2}{3u+\cfrac{2^2}{5u+\cfrac{3^2}{7u+\ddots}}}}
&&- \cfrac{4}{v+\cfrac{1^2}{3v+\cfrac{2^2}{5v+\cfrac{3^2}{7v+\ddots}}}}
\end{alignat*}
gde je $u=5$ i $v=239$.
Ta{\cv}nosti formule mo{\zv}e se jednostavno dokazati kompleksnom aritmetikom
$$
z=(5+i)^{16} / (239+i)^4 = -64
$$
Ta{\cv}ka $z$ ima koordinate $(-64,0)$ i sa $x$-osom zaklapa ugao od $\theta=180^\circ=\pi$ 
($\arg z=\pi$).
Izra\-{\cv}u\-na\-va\-ju\-{\cc}i \vr1 sa samo prva 4 {\cv}lana dobija se
\begin{align*}
\pi
& \approx\frac{53600}{16971} - \frac{1433464640}{85650012051}
= \frac{507390368614240}{161507372724169}\\
\noalign{\smallskip}
& \approx \underbrace{3.1415926}_{\hbox{\scriptsize ta{\cv}ne cifre}}\!10021\ldots
\end{align*}
broj $\pi$ sa 7 ta{\cv}nih cifara. Dovo{\lj}no je izra{\cv}unati navedenu formulu sa
prvih 18 {\cv}lanova (posled{\nj}i {\cv}lan je $\frac{17^2}{35x}$) da bi ta{\cv}nost dostigla 35 decimalnih cifara
\begin{align*}
\pi 
&\approx\textstyle{{1619055456573150058632544\over512630406240026708228595}}
-\textstyle{{61280517996953936280266854011343676617970098482609824\over3661532317204997134560556906873741209030782402728252845}}\\
\noalign{\smallskip}
&\approx\textstyle{{1540109464588823719896738887665292503532998573842247313242210592162332672
\over490232068383847267892878395373585882797677624279678806441294528302716687}}\\
\noalign{\smallskip}
&\approx 
\underbrace{3.14159265358979323846264338327950288}_{\hbox{\scriptsize ta{\cv}ne cifre}}\!07522\ldots
\end{align*}

Da bi postigao ovu ta{\cv}nost, nema{\cv}ko-holandski matemati{\cv}ar Ludolf (Ludolph van Ceulen) je
krajem 16.~i po{\cv}etkom 17.~veka,
koriste{\cc}i Arhimedovu metodu izra{\cv}unava{\nj}a povr{\sv}ina opisanih i upisanih pravilnih $2^n$-tougaonika u kru{\zv}nicu,
potro{\sv}io ve{\cc}i deo svog {\zv}ivota. {\Nj}emu u {\cv}ast broj $\pi$ se ponekad zove
{\sl Ludolfov broj}.

\subsection{$e$}

\def\raz#1/#2/{\frac{#1}{#2}}

Baza prirodnog logaritma
$$
e=2.
7182818284\,
5904523536\,
0287471352\,
6624977572\,
4709369996\,
\ldots
$$
mo{\zv}e bit izra{\cv}unata kao $e=\exp(1)$, pomo{\cc}u \vr2 za eksponencijalnu funkciju. 
Me{\dj}utim, postoji jednostavnija formula
$$
e=1+\frac{1}{\Kinf ii}
$$
Dovo{\lj}no je izra{\cv}unati prvih 10 konvergenata \vr2
$$
e\approx
2, 3, \raz8/3/, \raz30/11/, \raz144/53/, \raz280/103/, \raz5760/2119/, \raz45360/16687/, 
\raz44800/16481/, \raz3991680/1468457/
$$
da bi dobili 7 ta{\cv}nih decimala, $e_{10}\approx2.7182818$.
Konvergent 
$$e_{40}= 
\raz836313165329095177704251551336018791628800000000/
307662419905587585654556899376849633804439730767/
$$
je ta{\cv}an na 50 decimala.

\subsection{Borin broj}

\def\ofrac#1{\cfrac{1}{#1}}
\def\B{{\cal B}}
Ako broj
$$
\B=0.
4323320871\,
8590286890\,
9253793241\,
9999637051\,
1089687765\,
\ldots
$$
pretvorimo u \vr1, dobi{\cc}emo
\begin{align*}
\B &= \ofrac{\fbox2+\ofrac{\fbox3+\ofrac{\fbox5+\ofrac{\fbox7+\ofrac{\fbox{11}+\ofrac{\fbox{13}+\ddots}}}}}}
=\Kinf1{p_i}
\end{align*}
\vr1 {\cv}iji su imenioci {\it svi} {\sl prosti brojevi} $p$.
Naravno, zbog svoje prirode, ovaj broj ne mo{\zv}e biti ta{\cv}no izra{\cv}unat,
ali je zanim{\lj}ivo da postoji.


\section{Literatura}


%\printbibliography

\end{document}