\indexprologue{\noindent\normalsize
Ово је азбучни списак кључних речи и најбитнијих појмова из овог рада.
Поред сваког појма, {\sl искошеним цифрама\/} су исписани бројеви страна на којима се тај појам налази.
Наравно, нису приказане све стране, већ само где се налази дефиниција тог појма или где је битна
његова употреба.\par
Код верзије документа у електронском облику ради {\sl hyperlink\/} до наведене стране,
као и за све остале одреднице у раду.
За преузимање документа потребно је скенирати {QR} код који се налази на задњој корици рада.}

\begingroup
\footnotesize
\let\hp=\hyperpage
\def\hyperpage#1{\ifx0#1\hyperref[titlepage]{\lower2pt\hbox{\scalebox{0.5}\bcbook}}\else{\sl\hp{#1}}\fi}
\def\interskip{\vskip 3pt}
%\printindex

\begin{theindex}

  \item 2D, \hyperpage{11}
  \item 3D, \hyperpage{11}

  \indexspace

  \item 2, \hyperpage{6}, \hyperpage{14}
  \item \4, \hyperpage{14}
  \item 10, \hyperpage{6}

  \indexspace

  \item $x$-оса, \hyperpage{11, 12}
  \item $y$-оса, \hyperpage{11}
  \item $z$-оса, \hyperpage{11}

  \indexspace

  \item алгоритам, \hyperpage{8}, \hyperpage{26}
  \item антилогаритам, \hyperpage{3}, \hyperpage{17}, 
		\hyperpage{19}
  \item апсолутна вредност $\vert x\vert$, 
		\hyperpage{5}, \hyperpage{10, 11}, \hyperpage{17}
  \item аргумент, \hyperpage{3}
  \item асоцијативност, \hyperpage{11}

  \indexspace

  \item база, \hyperpage{3}
  \item Бенфордов закон, \hyperpage{12}
  \item Бернули, \hyperpage{7}
  \item бесконачност $(\infty)$, \hyperpage{3}
  \item бинарни логаритам, \hyperpage{6}
  \item Бригс, \hyperpage{9}
  \item бројна вредност, \hyperpage{7, 8}, \hyperpage{26}
  \interskip
  \item \BASIC, \hyperpage{26}

  \indexspace

  \item вектор, \hyperpage{11}
  \item верижни разломак, \hyperpage{8}
  \item вероватноћа, \hyperpage{12}
  \item версор, \hyperpage{11}
  \interskip
  \item {\sc  WikipediA}, \hyperpage{28}
  \item Wolfram MathWorld, \hyperpage{28}

  \indexspace

  \item Гаус, \hyperpage{8}
  \item геометријски низ, \hyperpage{23}
  \item график, \hyperpage{3}, \hyperpage{12}

  \indexspace

  \item декадни логаритам, \hyperpage{6}
  \item дефиниција, \hyperpage{3}
  \item децибел, \hyperpage{6}
  \item дигитрон, \hyperpage{9}
  \item Дирак, \hyperpage{14}

  \indexspace

  \item $\e$, \hyperpage{7}, \hyperpage{15}, \hyperpage{24}
  \item експонент, \hyperpage{7}
  \item експоненцијална функција, \hyperpage{7}
  \item $\exp$, \hyperpage{7}, \hyperpage{10, 11}
  \item епсилон $(\varepsilon)$, \hyperpage{8}, \hyperpage{26}
  \interskip
  \item ENIAC, \hyperpage{9}

  \indexspace

  \item збир, \hyperpage{11}
  \interskip
  \item \textsf{ZX Spectrum}, \hyperpage{26}

  \indexspace

  \item $i$, \hyperpage{10, 11}, \hyperpage{24}
  \item извод, \hyperpage{7}, \hyperpage{12}, \hyperpage{22}, 
		\hyperpage{24}
  \item имагинарна јединица, \hyperpage{24}
  \item интеграл, \hyperpage{12}
  \interskip
  \item iPhone, \hyperpage{9}

  \indexspace

  \item $j$, \hyperpage{11}
  \item јединични вектор, \hyperpage{11}
  \item једнакости, \hyperpage{4}
  \interskip
  \item {\sf  YouTube}, \hyperpage{28}

  \indexspace

  \item $k$, \hyperpage{11}
  \item квадратна једначина, \hyperpage{13--15}, 
		\hyperpage{17--19}
  \item кватернион, \hyperpage{11}
  \item количник, \hyperpage{4}
  \item компјутер, \hyperpage{9}, \hyperpage{26}
  \item комплексан број, \hyperpage{10}, \hyperpage{24}
  \item комплексна бесконачност $(\rsinfty)$, 
		\hyperpage{10}
  \item комутативност, \hyperpage{11}
  \item конвергент, \hyperpage{8}
  \item конјугована вредност $(\con z)$, 
		\hyperpage{11}
  \item корен $(\sqrt x)$, \hyperpage{14, 15}, \hyperpage{25}

  \indexspace

  \item лимес, \hyperpage{7}, \hyperpage{12}
  \item логаритам, \hyperpage{3}
  \item логаритмар, \hyperpage{9}, \hyperpage{25}
  \item логаритмовање, \hyperpage{10}, \hyperpage{15}
  \interskip
  \item $\ln$, \hyperpage{7, 8}, \hyperpage{10, 11}, \hyperpage{26}
  \item $\ln 10$, \hyperpage{8}
  \item $\ln 2$, \hyperpage{8}, \hyperpage{26}
  \item $\ln3$, \hyperpage{26}
  \item $\logten$, \hyperpage{6}, \hyperpage{23, 24}
  \item $\log_2$, \hyperpage{6}, \hyperpage{14}, \hyperpage{16}, 
		\hyperpage{23}

  \indexspace

  \item магнитуда, \hyperpage{23}
  \item максимум, \hyperpage{22}
  \item мантиса, \hyperpage{7}
  \item Mathematica, \hyperpage{27}
  \item матурски рад, \hyperpage{0}
  \item Меклорен, \hyperpage{10}
  \item Меклоренов ред, \hyperpage{7}, \hyperpage{10}
  \item минимум, \hyperpage{18}

  \indexspace

  \item највеће цело $\lfloor x\rfloor$, \hyperpage{7}, 
		\hyperpage{23}
  \item нзд, \hyperpage{16}
  \item Непер, \hyperpage{9}
  \item норма, \hyperpage{11}

  \indexspace

  \item Њуком, \hyperpage{12}

  \indexspace

  \item Ојлер, \hyperpage{7}, \hyperpage{12}
  \item Ојлерова формула, \hyperpage{10}, \hyperpage{24}
  \item основа, \hyperpage{3}

  \indexspace

  \item пи $(\pi)$, \hyperpage{10}, \hyperpage{24}
  \item Питагора, \hyperpage{17}
  \item Питагорина теорема, \hyperpage{17}
  \item Планк, \hyperpage{6}
  \item позор, \hyperpage{5}, \hyperpage{11}
  \item поларни запис, \hyperpage{10, 11}
  \item полураспад ($\thalf$), \hyperpage{7}, \hyperpage{23}
  \item правоугли троугао, \hyperpage{17}
  \item природни логаритам, \hyperpage{7}
  \item програм, \hyperpage{26}
  \item производ, \hyperpage{4}

  \indexspace

  \item реципрочна вредност, \hyperpage{4}, 
		\hyperpage{11}
  \item Рубикова коцка, \hyperpage{11}

  \indexspace

  \item скалар, \hyperpage{11}
  \item степен, \hyperpage{5}, \hyperpage{25}
  \item степен основе, \hyperpage{4}

  \indexspace

  \item таблице, \hyperpage{9}
  \item троугао $(\triangle )$, \hyperpage{17}
  \interskip
  \item \TeX, \hyperpage{27}
  \item Tik-Tok, \hyperpage{17}

  \indexspace

  \item факторијел $(n!)$, \hyperpage{4}, \hyperpage{7}, 
		\hyperpage{15}
  \item Фибоначијев низ, \hyperpage{12}
  \item формула, \hyperpage{7, 8}
  \item фуснота, \hyperpage{10, 11}, \hyperpage{13}, 
		\hyperpage{23}

  \indexspace

  \item Хамилтон, \hyperpage{11}

  \indexspace

  \item шибер, \hyperpage{9}
  \item што је требало доказати $(\square )$, 
		\hyperpage{14}, \hyperpage{22, 23}

\end{theindex}


\endgroup

\input dayofweek

\def\twodig#1{\ifnum#1>9\else0\fi\number#1}
\def\datum{\twodig\day.\twodig\month.\number\year.}
\newcount\hour \hour=\time \divide\hour by 60
\newcount\minute \minute=-\hour \multiply\minute by 60
\advance\minute by \time 
\def\vreme{\twodig\hour:\twodig\minute}

\par\vfill

% \rightline{\eightssi--- Luka S. Ne{\sv}i{\cc},\enspace\eightss (\the\year)}
\rightline{\footnotesize\textsf{--- \href{mailto:luka.s.nesic@gmail.com}{\textit{Лука С. Нешић}},\enspace\DayOfWeek~\datum~\vreme}}
