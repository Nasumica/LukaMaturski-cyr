\subsubsection{Четири четворке}

\def\4{{\color{red}\bf4}}\index{4@\4}

\zadatak
Доказати да сваки природан број $n\in\Nset$, може бити представљен са \4 броја \4,
помоћу логаритамске функције и квадратног корена\index{корен $(\sqrt x)$}
$$
n=\log_{\sqrt\4/\4}\left(\log_\4 \underbrace{\sqrt{\sqrt{\cdots\sqrt\4}}}_{\text{$n$ корена}}\right).
$$

\vskip-12pt
\resenje
Како је
$$
\frac{\sqrt \4}{\4}=\frac12
\qquad\text{и}\qquad
\underbrace{\sqrt{\sqrt{\cdots\sqrt \4}}}_{\text{$n$ корена}}=\4^{(1/2)^n},
$$
израз може бити упрошћен
\begin{align*}
% \noalign{\vskip-3pt}
\log_{\sqrt\4/\4}\left(\log_\4 \underbrace{\sqrt{\sqrt{\cdots\sqrt\4}}}_{\text{$n$ корена}} \right)
&=\log_{1/2}\left(\log_\4 \4^{(1/2)^n}\right),\\
\intertext{где из једнакости за логаритам степена основе \eqref{eq:powb}, следи}
&=\log_{1/2}(1/2)^n\\
&=\ram{n}.
\end{align*}

\def\2{{\it2}}\index{2}\QEDidx
\def\dlog{\mathop{\it\ell\mskip -0.5\thinmuskip og}\nolimits_\2}
\dodatak\begingroup
Давно је у једном часопису постављен сличан задатак: 
да се са што мање истих бројева,
користећи било коју математичку функцију, представи 
сваки природан број $n$.
Решио га је нобеловац \idx{Дирак} (Paul Dirac) 
са~3~броја~\2, чије оригинално
решење изгледа
$$
-\dlog\dlog\sqrt{\cdots n\cdots\sqrt\2\index{log2@$\log_2$}}
=-\dlog\dlog\2^{\2^{-n}}
=-\dlog\2^{-n}=n.
\eqno{\QED}$$
\endgroup
