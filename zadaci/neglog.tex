\subsubsection{ln($-$\textit{z})}

\zadatak 
\У скупу комплексних бројева\index{комплексан број} $\Cset$, ако знамо $\ln z$,
колико је $\ln(-z)$?

\resenje 
Како је у комплексној равни $-z$ једнако $z$ заротирано
око координатног почетка
за угао од $180^\circ=\pi$, добијамо\index{пи $(\pi)$}
$$
\ln (-z) = \ram{\ln z +i\pi}.
$$ 
Ако проверимо, из Ојлерове једнакости \eqref{eq:zeuler}, добијамо
$$
\e^{\ln(-z)} = \e^{\ln z+i\pi} = \e^{\ln z}\cdot \e^{i\pi}=z \cdot (-1)=-z.
$$

\dodatak Хммм \dots, трик са ротацијом није баш потпуно тачан: добили бисмо $-z$ и за угао $-\pi$,
па би било $\ln(-z)=\ln z-i\pi$, што је такође тачно;
у ствари, тачно је за било који угао $\pi+2k\pi$ где је $k\in\Zset$ цео број. Одавде би следило да је
$$
\ln(-z) = \ln z + i(\pi+2k\pi),\ k\in{\mathbb Z}.
$$

И сама формула \eqref{eq:cln}, $\ln z=\ln\rho + i\theta$, представља само {\sl главну грану\/}
комплексног логаритма, који је некаква врста 4D спирале. Потпуна формула би била\index{ро $(\rho)$}\index{тета $(\theta)$}
\begin{equation}
    \okvir{\ln z=\ln\rho + i(\theta + 2k\pi),\ k\in\Zset} .
\end{equation}