\subsubsection{Аналогни квадратни корен}\label{sssec:sibersqrt}

\zadatak
Објасни начин за одређивање вредности
$\sqrt x$\index{корен $(\sqrt x)$} логаритмаром\index{логаритмар}.

\resenje
\Уз мало вежбе, квадратни корен можемо директно читати са логаритмара ако  
{\sl у глави\/} извршимо
дељење са 2 и, по потреби, сабирање са $0\.5$
што је врло једноставно јер се ради о бројевима између 0 и 1 са највише 3 децимале. 
Како је
$$
\sqrt x=10^{\frac12\log x},
$$
потребно је прочитати вредност
$\log x$, 
а онда, за двоструко мању вредност од ње, прочитати вредност
$10^y$. На пример,
за израчунавање вредности $\sqrt{5\.3}$,
читамо да је $\log(5\.3)\approx0\.724$~($\color{red}\downarrow$), потом, за
$y=0\.724/2=0\.362$~($\color{red}\uparrow$), читамо вредност $10^y$ и
добијамо $\sqrt{5\.3}\approx 2\.3$  ($2\.3^2=5\.29$).
За $\sqrt{53}$ треба у $y$ додати још $0\.5$ тако да ће бити $y=0\.362+0\.5=0\.862$~($\color{magenta}\Uparrow$), 
одакле је $\sqrt{53}\approx7\.28$ ($7\.28^2=52\.9984$).
$$
\def\hair{\nit{0.36213793480039452281649614581363}\nit{0.72427586960078904563299229162726}\nit{0.86213793480039452281649614581363}}
\def\extra{\ruler{0.36213793480039452281649614581363}{$\uparrow$}\ruler{0.72427586960078904563299229162726}{$\downarrow$}\ruler{0.86213793480039452281649614581363}{$\color{magenta}\Uparrow$}}
\logaritmar0
$$
Наравно, $\sqrt{530}$ се рачуна као $10\sqrt{5\.3}$, или $\sqrt{0\.53}=\frac{1}{10}\sqrt{53}$.
