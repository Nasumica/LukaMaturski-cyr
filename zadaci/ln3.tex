\subsubsection{ln 3}\label{sssec:ln3}\index{ln3@$\ln3$}
 
\zadatak
U {\cv}ast Nepera i Brigsa,
pomo{\cc}u postupka \eqref{eq:alg}\index{algoritam} sa 
strane~\pageref{eq:alg},
izra{\cv}unaj {\sl pe{\sv}ke\/} pribli{\zv}nu vrednost $\ln 3$\index{brojna vrednost}
u~5 koraka. Za upore{\dj}iva{\nj}e, ta{\cv}na vrednost je
$$
\ln3=1\.
0986122886\,
6810969139\,
5245236922\,
5257046475\,\ldots
%1639157/1492025=1.098612288668
$$

\def\step#1{\par\indent\leavevmode
  Korak~{\it#1}.\kern2em\relax}

\resenje
Za $x=3$ bi{\cc}e $r=(x-1)/(x+1)=1/2$. U nultom koraku postav{\lj}amo po{\cv}etne vrednosti:

\smallskip

\step0 $k=1,\quad p=2r=1,\quad q=r^2=1/4,\quad a=p=1,\quad y=a=1$.

\smallskip

\noindent Slede koraci iteracije --- pove{\cc}amo $k$ za 2, pomno{\zv}imo $p$ sa $q$,
{\cv}lan sume $a$ postaje $p/k$, koga dodajemo u rezultat $y$:

\smallskip

\step1 $k=3,\quad p=1/4,\quad a=1/12,\quad y=13/12$;
\step2 $k=5,\quad p=1/16,\quad a=1/80,\quad y=263/240$;
\step3 $k=7,\quad p=1/64,\quad a=1/448,\quad y=7379/6720$;
\step4 $k=9,\quad p=1/256,\quad a=1/2304,\quad y=88583/80640$;
\step5 $k=11,\quad p=1/1024,\quad a=1/11264,\quad y=3897967/3548160$.

\smallskip

\noindent Rezultat je
$$
\ln3\approx\frac{3897967}{3548160}=
\ram{1.098588\ldots} 
% \ram{1\.0986},
$$
{\sv}to nije lo{\sv}e za samo 5 koraka, jer je apsolutna gre{\sv}ka oko $2\.4\puta10^{-5}$.
Ali, mo{\zv}e bo{\lj}e~\dots

\dodatak
Ako ve{\cc} imamo precizno izra{\cv}unatu vrednost $\ln2$\index{ln2@$\ln 2$}, onda je bo{\lj}e ra{\cv}unati $\ln3$ kao $\ln(3/4)+2\ln2$,
jer {\cc}e u postupku, umesto $r=1/2$, biti $r=(3/4-1)/(3/4+1)=-1/7$, 
odnosno, umesto $q=1/4$, bi{\cc}e $q=1/49$,
{\sv}to dovodi do mnogo br{\zv}eg izra{\cv}unava{\nj}a. U~istom broju koraka bismo dobili
$$
\ln\frac34\approx
-\ff2/7, -\ff296/1029, -\ff72526/252105, -\ff24876448/86472015, 
-\ff522405418/1815912315, -\ff281576520392/978776737785,
$$
gde posled{\nj}i razlomak ima gre{\sv}ku od oko $1\.6\puta10^{-12}$,
{\sv}to je vi{\sv}e od dvostruko ta{\cv}nih cifara.
Kada mu (sa strane~\pageref{eq:ln2}) dodamo $2\ln2$, dobi{\cc}emo
$$
\ln3\approx 2\ln2 -\frac{281576520392}{978776737785} =
{\,\underbrace{\!1\.0986122886\,6\!}_{\hbox{\scriptsize 12 ta{\cv}nih cifara}}972615081\,\ldots}
$$

Uop{\sv}teno, postupak je najbr{\zv}i ako ra{\cv}unamo $\ln x=\ln(x/2^n)+n\ln2$,\index{ln@$\ln$}
gde biramo $n$ takvo da $x/2^n$ bude {\sv}to bil{\zv}e 1, odnosno, da $q$ bude najma{\nj}e mogu{\cc}e (vidi \idx{program}).
$$
\slika{\hbox to \textwidth{\includegraphics[width=\zxscreen]{ln-prog.png}\hss
  \includegraphics[width=\zxscreen]{ln-run.png}}}{\href{https://fuse-emulator.sourceforge.net/}{\textsf{ZX Spectrum}} \href{\GitHubRaw ln.z80}{\BASIC} program.}
$$\index{kompjuter}\index{ZX Spectrum@\textsf{ZX Spectrum}}\index{BASIC@\BASIC}\index{epsilon $(\varepsilon)$}
