\subsubsection{Полураспад јода}

\def\iso#1-#2{\vphantom{\rm#2}^{#1}{\rm#2}}

\zadatak 
Ако имамо $63\um g$ изотопа јода $\iso 131-I$, а знамо да смо пре 11 дана имали $163\um g$, које је време
полураспада\index{полураспад ($\thalf$)} овог изотопа? (Користи природни логаритам.)

\resenje
Из формуле \eqref{eq:halftime} на страни \pageref{eq:halftime}, следи да се време полураспада
може израчунати\index{log2@$\log_2$}
$$
\thalf
=\frac{t}{\logtwo(m_0 / m_t)}
=\frac{t\ln2}{\ln(m_0 / m_t)}
=\frac{11\ln2}{\ln(163/63)}
\approx\ram{8\.02\um{дана}}.
$$
