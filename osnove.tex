\section{Најчешће логаритамске основе}

\subsection{Основа 10}\index{10}\index{log10@$\logten$}\index{декадни логаритам}

У инжењерству се најчешће користи логаритам са основом 10,
зове се {\sl декадни\/} или {\sl заједнички\/} логаритам, и пише се
$$
y=\logten x.
$$
% или, скраћено,
% $$
% y=\lg x.
% $$
Понекад се може видети и само
$$
y=\log x,
$$
без навођења основе, али треба обратити пажњу на контекст.
Ако је неки инжењерски текст у питању, нејвероватније се мисли на основу 10.

Декадни логаритам је погодан и када се користи, такозвани
{\sl научни\/} или {\sl инжењерски\/} запис броја.
На пример, {\sl Планкова константа\/} (Max Planck)\index{Планк} износи
$$
h=6\.62607015\puta 10^{-34}\um{J/Hz}
$$
која има декадни логаритам
$$
\logten h=\logten(6\.62607015) - 34.
$$

У физици се за мерење нивоа сигнала или звука користи јединица {\sl бел\/}~(B), али је чешће
у практичној употреби 10 пута мања јединица {\sl децибел\/}~(dB)\index{децибел}, односно, $1\um{B}=10\um{dB}$. \
Ниво сигнала $L$, који зависи
од односа измерене снаге $P$ и референтне снаге $P_0$, изражен у децибелима износи
$$
L=10\logten\left(\frac{P}{P_0}\right)\um{dB}.
$$
Како се у акустици узима да је референтна снага $P_0=10^{-12}\um W$, могло би се писати
да је ниво звука у децибелима
$$
L=10\logten(P)-120.
$$

Нормалан говор је око $50\um{dB}$,
звук мотора млазног авиона при полетању је $150\um{dB}$, 
%raketa Saturn~V oko $220\um{dB}$,
а смртоносан је звук од $240\um{dB}$ и више.
Звучни топ {\sf Genasys LRAD} има ниво звука око $160\um{dB}$,
што значи да је $10^{11}$ пута моћнији од говора.
%Naravno, treba uzeti u obzir da nivo opada sa kvadratom rastoja{\nj}a.

Декадни логаритам се користи и за одређивање јачине земљотреса: $M=\logten I$, или 
pH вредности: ${\rm pH}=-\logten[{\rm H}^+]$.

\subsection{Основа 2}\index{2}\index{log2@$\log_2$}\index{бинарни логаритам}

\def\lb{\mathop{\rm lb}}
\def\bits{{\it bits}}
\def\mant{\textit{mantissa\/}}%
\def\expo{\textit{exponent\/}}%
\def\znak{{\it знак}}%

У информатици се често користи логаритам са основом 2, који се зове
{\sl бинарни\/} логаритам, и пише се
$$
y=\logtwo x.
$$
% ili, skra{\cc}eno,
% $$
% y=\lb x.
% $$
Користи се у комбинаторици, као за одређивање {\sl количине информација},
односно, потребног броја битова меморије за смештање неког податка.
Ако се зна да ће у меморију бити уписивани цели бројеви од
0 до $n$, онда је потребно резервисати
\begin{equation}\label{eq:bits}
  \bits = \lfloor\logtwo(n)\rfloor+1
\end{equation}
битова меморије, где $\lfloor x\rfloor$\index{највеће цело $\lfloor x\rfloor$} 
представља {\sl највећи сео број који је мањи или једнак} $x$
(изговара се \navod{највеће цело од $x$}).
На пример, ако ће у одређеној меморији највећи  
број бити милион, онда је за то потребно резервисати
$$
\bits=\lfloor\logtwo(1\,000\,000)\rfloor+1=\lfloor 19\.9315685693\rfloor+1 = 19+1=20
$$
битова меморије.
% Највећи број који може стати у ових резервисаних 20 битова меморије је бинарни број који има 20 јединица 
% и износи 
% $$
% (1111\,1111\,1111\,1111\,1111)_2=
% 2^{20}-1=1\,048\,575.
% $$

Како су и реални бројеви у меморији представљени као уређени парови бинарних бројева у облику
$x=(\mant,\expo)$\index{мантиса}\index{експонент}, са значењем
$$
x=\mant\puta2^\expo,
$$
бинарни логаритам би био израчунат као
$$
\logtwo(x)=\logtwo(\mant)+\expo,
$$
ако је $\mant>0$, иначе је недефинисан.

\smallskip

Бинарни логаритам се користи и у атомској физици.
Време {\sl полураспада\/} $\thalf$\index{полураспад ($\thalf$)}\
је време потребно да се распадне половина језгара атома неке материје.
Ако имамо почетан број језгара
$N_0$ и број језгара $N_t$ након времена $t$, њихов однос
се може представити формулом
\begin{equation}
\label{eq:halftime}
\frac{N_0}{N_t}=2^{t/\thalf}\sledi \frac{t}{\thalf}=\logtwo\left( \frac{N_0}{N_t} \right).
\end{equation}
Ова формула се користи и за одређивање старости стена или фосила.


%\newpage

\subsection{Основа \e}\index{е@$\e$}\index{ln@$\ln$}\index{природни логаритам}

Ову логаритамску основу је открио Јакоб \idx{Бернули} (Jacob Bernoulli) када је
проучавао {\sl сложену камату\/} и доказао да {\sl континуална\/} сложена камата
тежи константи
\begin{equation}\label{eq:elim}
\e=\lim_{n\to\infty}\left(1+\frac1n\right)^{\!n}\!,\index{лимес}
\end{equation}
али је тек \idx{Ојлер} (Leonhard Euler)
одредио њену тачну вредност и дао јој име.
Логаритам за ову основу се зове
{\sl природни\/} логаритам ({\sl logarithmus naturalis\/})
и пише се
$$
\ln x=\log_\e x.
$$
% (Изговара се \navod{ел-ен од $x$}.)

\def\ep{\hphantom{!}}%
\def\rf#1!{\frac1{\hphantom{!}#1!}}%
Антилогаритам је {\sl експоненцијална\/} функција\index{експоненцијална функција} $\e^x=\exp(x)$\index{exp@$\exp$}, 
која је позната по томе што је то
једина функција чији је први
\idx{извод} једнак самој функцији: $\exp'(x)=\exp(x)$. 
Бројна вредност се може израчунати формулом\footnote{Како је $\e^0=1$, а из формуле \eqref{eq:exp} се добија
$\e^0=0^0/0!$ (остали чланови реда су 0), следи да је $0^0=0!=1$, 
у чему је математички свет је прилично подељен. Програмерски није, $\forall x\colon x^0={\bf1}$.}
\begin{equation}
\label{eq:exp}\index{формула}
\okvir{\e^x=\exp(x)=\sum_{n=0}^\infty\frac{x^n}{n!}}.
\end{equation}\index{факторијел $(n"!)$}\index{Меклоренов ред}%
Ако ставимо да је $x=1$,
\idx{бројна вредност} основе природног логаритма $\e$ се може одредити
\begin{equation}\label{eq:e}
\begin{split}
\e
&=\rf0!+\rf1!+\rf2!+\rf3!+\rf4!+\rf5!+\cdots\\
\noalign{\smallskip}
&=2\.
7182818284\,
5904523536\,
0287471352\,
6624977572\,\ldots
\end{split}
\end{equation}
са жељеном тачношћу.
