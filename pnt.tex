\subsection{Teorema prostih brojeva}

Jo{\sv} jedno mesto gde se pojav{\lj}uje prirodni logaritam u {\sl teoriji brojeva}\index{teorija brojeva} je,
takozvana, {\sl teorema prostih brojeva\/} (PNT), kojom se je bavio \idx{Gaus} kada je
imao samo 15--16 godina.

% \smallskip

Ako funkcija $\pi(n)$ ima vrednost {\sl ukupan broj prostih bro\-je\-va ne ve{\cc}ih od~$n$}
(na primer, $\pi(10)=4$ jer do 10 postoje 4 prosta broja: 2, 3, 5, i 7), onda va{\zv}i\index{limes}
$$
\lim_{n\to\infty}\frac{\pi(n)\ln n}{n} = 1.\index{limes}
$$
Posledica ove teoreme je da je $n$-ti \idx{prost broj} $p_n$, za veliko $n$, otprilike
$$
p_n\sim n\ln n.
$$

